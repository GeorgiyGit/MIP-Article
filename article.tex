% Metódy inžinierskej práce

\documentclass{article}

%\usepackage[T1]{fontenc}
\usepackage[IL2]{fontenc} % lepšia sadzba písmena Ľ než v T1
\usepackage[utf8]{inputenc}
\usepackage{graphicx}
\usepackage{url} % príkaz \url na formátovanie URL
\usepackage{hyperref} % odkazy v texte budú aktívne (pri niektorých triedach dokumentov spôsobuje posun textu)
\usepackage{enumitem}  % For adjusting list spacing
\usepackage{indentfirst}
\usepackage{amsmath}
\usepackage{float}
\usepackage{tabularx}
\usepackage{array}
\usepackage{booktabs}
\usepackage{multirow}
\usepackage{multicol}
\usepackage{xcolor}

\usepackage{cite}
%\usepackage{times}
\usepackage{caption}
\captionsetup{labelsep=period}

\pagestyle{headings}

\title{Content-based filtering\thanks{Semester project in the subject Methods of engineering work. Year 2024/25, supervision: Pavol Baťalík}}

\author{Sladkovskyi Heorhii\\[2pt]
	{\small Slovak University of Technology in Bratislava}\\
	{\small Faculty of Informatics and Information Technologies STU in Bratislava}\\
	{\small \texttt{xsladkovskyi@stuba.sk}}
	}

\date{\small 3. october 2024}



\begin{document}

\maketitle
\tableofcontents

%\section*{Reference Papers}\label{sec:reference_papers}


DL
\begin{enumerate}[noitemsep, topsep=1px]
    \item \href{https://dl.acm.org/doi/10.1145/3610548.3618189}{Content-based Search for Deep Generative Models} 
    \item \href{https://dl.acm.org/doi/10.1145/2611286.2611300}{Content-based filtering discovery protocol (CFDP): scalable and efficient OMG DDS discovery protocol}
    \item \href{https://dl.acm.org/doi/10.1145/2661829.2661940}{Social Book Search Reranking with Generalized Content-Based Filtering}
    \item \href{https://dl.acm.org/doi/10.1145/2501025.2501036}{Customized reviews for small user-databases using iterative SVD and content based filtering}
    \item \href{https://dl.acm.org/doi/10.1145/1277741.1277903}{A multi-criteria content-based filtering system}
    \item \href{https://dl.acm.org/doi/10.1145/1015330.1015394}{Unifying collaborative and content-based filtering}
    \item \href{https://dl.acm.org/doi/10.1145/544220.544341}{Content-based filtering and personalization using structured metadata}
    \item \href{https://dl.acm.org/doi/10.1145/2492517.2500283}{Recommender system by grasping individual preference and influence from other users}
\end{enumerate}

Springer link
\begin{enumerate}[noitemsep, topsep=1px]
    \item \href{https://link.springer.com/chapter/10.1007/978-981-97-0538-2_3}{Collaborative Filtering and Content-Based Systems}
    \item \href{https://link.springer.com/chapter/10.1007/978-3-031-35078-8_37}{Movie Recommendation Using Content-Based and Collaborative Filtering Approach}
    \item \href{https://link.springer.com/chapter/10.1007/978-3-031-59707-7_9}{Sentiment Analysis and Innovative Recommender System: Enhancing Goodreads Book Discovery Using Hybrid Collaborative and Content Based Filtering}
    \item \href{https://link.springer.com/chapter/10.1007/978-3-031-34622-4_13}{Integrated Music Recommendation System Using Collaborative and Content Based Filtering, and Sentiment Analysis}
    \item \href{https://link.springer.com/chapter/10.1007/978-981-99-2287-1_50}{Personalized Recommendation of Literature Resources in University Library Based on Abstract Content Filtering Algorithm}
    \item \href{https://link.springer.com/chapter/10.1007/978-981-16-2594-7_47}{Job Recommendation System Using Content and Collaborative-Based Filtering} 
    \item \href{https://link.springer.com/chapter/10.1007/978-981-19-7660-5_8}{Development of Hybrid Personalized E-commerce Using Collaborative Filtering and Content-Based Filtering for South Cartel Clothing Company}
    \item \href{https://link.springer.com/chapter/10.1007/978-3-031-19958-5_98}{A Hybrid Hotel Recommendation Using Collaborative, Content Based and Knowledge Based Approach}
\end{enumerate} %TODO : temp, only for development
\begin{abstract}
    In recent years, recommendation systems have become an essential part of different platforms from e-commerce to the majority of streaming platforms. Content-based filtering is a widely used approach that recommends new items based on user previous interactions and preferences. This article will consider the advantages and disadvantages of content-based filtering, different algorithms for CBF, the connections between this method and other types of recommendation algorithms, the advantages and disadvantages of each algorithm and in which situations it is better to use each of them. Additionally, the article provides an analysis of popular Content-based filtering methods efficiency based on a dataset of books and user interactions.
\end{abstract}
\newpage %TODO : temp, only for development
\section{Introduction}\label{sec:introduction}
Nowadays, internet resources are overfilled with large amounts of information, within which the user needs to be provided with relevant data to his interests. Without the help of the environment, users could not analyze this mass of information, which reduces the amount of processed data\cite{InternetBigData}.

Recommendation algorithms become a solution to this problem, because they make it possible to select individual content for each user, depending on their actions or the actions of other users. This approach improves the quality of information provided to the user making it more relevant to user preferences\cite{IMRSUCBF}.

Recommendation systems are used in the majority of social networks. Youtube uses different algorithms to recommend content that best matches users' preferences. Linkedin provides flexible jobs searching, Netflix suggests movies based on previous user actions\cite{CBF_In_Social_Networks}.

Content-based filtering (CBF) is an instrument, which is based on finding similarities between user and content. At the same time, the user`s previous actions as well as their profile and the other information, can be taken into account. CBF can be used with different types of data, such as text, audio and images\cite{van2000using}.

Due to the flexibility of Content-based filtering, it is highly used in many approaches. With the use of this approach, recommendation systems provide more relevant data to user preferences, because this is the main idea of CBF. Different systems use different algorithms and types of Content-based filtering, based on the needs and structure of provided data. %TODO : provide a cite if needed

Content-based filtering could be a part of hybrid recommendation systems, which use multiple recommendation approaches to select relevant data and minimize disadvantages of each method. In this case, not only individual properties of CBF, but also a new combination of of this method and some others must be considered\cite{hybrid_systems}.

\section{Algorithms for content-based filtering}\label{sec:cbf_algorithms}
Content-based filtering could be used in a variety of contexts. Depending on the needs of a recommendation system we could have different implementations of CBF. Depending on data type, it comonly used models include Term frequency–inverse document frequency (TF-IDF)\cite{TF_IDF}, Naive Bayes classifier \cite{Naive_classifier}, k-Nearest Neighbors or neural networks. The most popular and widely methods are based on vector 

\subsection{Decision-tree method}
Decision-tree could be used not only with content-based filtering, but also with other recommendation methods. It is not really popular due to several reasons, but could have advantages in specific situations. Decision-tree method relies on inductive learning, therefore we firstly should have some data from the user (his preferences or previous actions). After that, we could use algorithms like C4.5\cite{C4_5} or newer version C5.0. Given algorithm will produce a decision tree based on user preferences and it will be used as a user profile. The example of a decision tree is shown in Figure~\ref{fig:decision_tree}\cite{Decision_Tree}.

After construction of the decision tree, we can classify new items, with which the user has not yet interacted. If the item has similar categories (this approach could be used only if items have some sort of categorization) to those which are preferable to user profile, the output of the decision tree for these items would be abstract “Like”. This will mean that we could recommend a new item to a user with a high probability that this item will interest the user. The decision tree can be reconstructed during the process of interacting with new content which will improve the recommendation results.

However, this algorithm has some disadvantages. Firstly, the computational complexity of C4.5 during the training data is $\mathbf{O(m^2)}$. For a small amount of data it wouldn't be a problem, but if the user is interacting with thousands of items, rebuilding the decision tree won't be a trivial task. Computational cost during the recommendation process is $\mathbf{O(n)}$, which also could be a problem with high amounts of presented items\cite{Decision_Tree}.
\begin{center}
    \begin{figure}[H]
    \includegraphics[width=\textwidth]{figures/diagrams/article_decision_tree.pdf}
    \caption{Example of decision-tree}
    \label{fig:decision_tree}
    \end{figure}
\end{center}

\subsection{K-Nearest Neighbor method}
K-Nearest Neighbor (KNN) belongs to vector space methods that are used in N dimensional environments (see section 3 for more details). It is highly used and has better results than decision-tree in the majority of data sets\cite{KNN}.

But before using KNN, we should collect data from user interactions (e.g. user likes) into a user-item matrix. The example of the user-item matrix is shown in Figure 2. In example, user has different amount of 

\begin{table}[h]
    \centering
    \begin{tabular}{|c|c|c|c|c|}
        \hline
        \textbf{Users} & \textbf{Movie 1} & \textbf{Movie 2} & \textbf{Movie 3} & \textbf{Movie 4} \\        
        \hline
        User 1 & 3 & 0 & 1 & 2\\
        \hline
        User 2 & 0 & 5 & 5 & 3\\
        \hline
        User 3 & 5 & 1 & 1 & 2\\
        \hline
        User 4 & 0 & 0 & 5 & 0\\
        \hline
    \end{tabular}
    \caption{User-item matrix}\label{tab:user_item_matrix}
\end{table}

\begin{table}[h]
    \centering
    \begin{tabular}{|c|c|c|c|c|}
        \hline
        \textbf{Genres} & \textbf{Movie 1} & \textbf{Movie 2} & \textbf{Movie 3} & \textbf{Movie 4} \\        
        \hline
        Action  & 1 & 0 & 0 & 1\\
        \hline
        Romance & 0 & 1 & 1 & 0\\
        \hline
        Comedy  & 1 & 1 & 1 & 0\\
        \hline
        Horror  & 0 & 0 & 0 & 1\\
        \hline
    \end{tabular}
    \caption{Genre-movie matrix}\label{tab:genre_movie_matrix}
\end{table}

\newcommand{\cosinesimilarity}{%
    \begin{equation}
    \text{similarity} = \cos(\theta) = \frac{\vec{A} \cdot \vec{B}}{||\vec{A}|| \times ||\vec{B}||} = \frac{\sum_{i=1}^{n} A_i B_i}{\sqrt{\sum_{i=1}^{n} A_i^2} \times \sqrt{\sum_{i=1}^{n} B_i^2}}
    \end{equation}
}
\begin{table}[h]
    \centering
    \begin{tabular}{|c|c|c|c|c|}
        \hline
         & Movie 1 & Movie 2 & Movie 3 & Movie 4 \\        
        \hline
        Movie 1 &1		&0.5	&0.5	&0.5\\
        \hline	
        Movie 2 &0.5	&1      &1		&0	\\
        \hline
        Movie 3 &0.5	&1		&1		&0	\\
        \hline
        Movie 4 &0.5	&0		&0		&1	\\
        \hline
    \end{tabular}
    \caption{Cosine similarity matrix}\label{tab:cosine_similarity_matrix}
\end{table}
\section{Data representation in content-based filtering}\label{sec:cbf_data_rep}

\subsection{Data extraction} %TODO

\subsection{Data unification using embeddings} %TODO

\subsection{Advantages of embeddings in CBF} %TODO
\section{Advantages and disadvantages of content-based filtering}\label{sec:cbf_advantages_disadvantages} %TODO : could be rewrited to usecases of CBF

\subsection{Advantages} %TODO

\subsection{Limitations} %TODO

 %TODO : temp, could be removed
\section{Comparison of CBF with other recommendation methods}\label{sec:cbf_comparison}

\section{Advantages and disadvantages} %TODO : temp, could be removed

\subsection{Collaborative filtering vs. content-based filtering} %TODO : rewrite name

\subsection{Hybrid Models} %TODO
\section{Use cases}\label{sec:usecases} % TODO : change the name

\subsection{Media streaming platforms} %TODO

\subsection{News} %TODO

\subsection{E-commerce} %TODO
\section{Conclusion}\label{sec:conclusion}
This study presents a comparative analysis of Content-Based Filtering (CBF) algorithms, specifically focusing on Decision Tree (C4.5) and K-Nearest Neighbor (KNN) with different distance metrics, using the Book Crossing dataset. The analysis was performed under two output settings: 10 categories (ratings 1-10) and 3 categories (grouped ratings), with F1 Score and Root Mean Square Error (RMSE) as the performance metrics.

The results demonstrate that dividing ratings into 3 categories significantly improve the predicted performance of all algorithms. From all tested methods, KNN with Manhattan distance achieved the highest F1 Score (0.863) and the lowest RMSE (0.287) in the 3 category output. Similarly, Decision tree (C4.5) performed well, having F1 Score (0.858) and RSME (0.3).

On the other hand, KNN with Minkowski distance have results not better than random prediction. Additionally, KNN with Hellinger, Minkowski and Levenshtein distances have weaker performance for the used dataset. Furthermore, the analysis of computational efficiency revealed that Decision Tree (C4.5), despite a higher build time, offered faster prediction times, making it advantageous for real-time recommendation. On the other hand, KNN algorithm has smaller trained time, but slower predicting especially with complex distance metrics.

Considering overall Content-Based Filtering, it is scalable to a large number of customers,  can recommend specific unpopular items and recommends items immediately. On the other hand, there is a “cold start” problem and CBF couldn't generate relevant content to non-active users. Hybrid system reduces disadvantages of a Content-Based Filtering combining this method with the others.

\nocite{*} %TODO : temp

\bibliography{bibliography/references}
\bibliographystyle{unsrt}
\end{document}
