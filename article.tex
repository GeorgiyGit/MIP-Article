% Metódy inžinierskej práce

\documentclass{article}

\usepackage[slovak]{babel}
%\usepackage[T1]{fontenc}
\usepackage[IL2]{fontenc} % lepšia sadzba písmena Ľ než v T1
\usepackage[utf8]{inputenc}
\usepackage{graphicx}
\usepackage{url} % príkaz \url na formátovanie URL
\usepackage{hyperref} % odkazy v texte budú aktívne (pri niektorých triedach dokumentov spôsobuje posun textu)
\usepackage{enumitem}  % For adjusting list spacing

\usepackage{cite}
%\usepackage{times}

\pagestyle{headings}

\title{Content-based filtering\thanks{Semester project in the subject Methods of engineering work. Year 2024/25, supervision: Pavol Baťalík}}

\author{Sladkovskyi Heorhii\\[2pt]
	{\small Slovak University of Technology in Bratislava}\\
	{\small Faculty of Informatics and Information Technologies STU in Bratislava}\\
	{\small \texttt{xsladkovskyi@stuba.sk}}
	}

\date{\small 3. october 2024}



\begin{document}

\maketitle
\tableofcontents

\section*{Reference Papers}\label{sec:reference_papers}


\begin{enumerate}[noitemsep, topsep=1px]
    \item \href{https://dl.acm.org/doi/10.1145/3610548.3618189}{Content-based Search for Deep Generative Models} 
    \item \href{https://dl.acm.org/doi/10.1145/2611286.2611300}{Content-based filtering discovery protocol (CFDP): scalable and efficient OMG DDS discovery protocol}
    \item \href{https://dl.acm.org/doi/10.1145/2661829.2661940}{Social Book Search Reranking with Generalized Content-Based Filtering}
    \item \href{https://dl.acm.org/doi/10.1145/2501025.2501036}{Customized reviews for small user-databases using iterative SVD and content based filtering}
    \item \href{https://dl.acm.org/doi/10.1145/1277741.1277903}{A multi-criteria content-based filtering system}
    \item \href{https://dl.acm.org/doi/10.1145/1015330.1015394}{Unifying collaborative and content-based filtering}
    \item \href{https://dl.acm.org/doi/10.1145/544220.544341}{Content-based filtering and personalization using structured metadata}
    \item \href{https://dl.acm.org/doi/10.1145/2492517.2500283}{Recommender system by grasping individual preference and influence from other users}
\end{enumerate}

Springer link
\begin{enumerate}[noitemsep, topsep=1px]
    \item \href{https://link.springer.com/chapter/10.1007/978-981-97-0538-2_3}{Collaborative Filtering and Content-Based Systems}
    \item \href{https://link.springer.com/chapter/10.1007/978-3-031-35078-8_37}{Movie Recommendation Using Content-Based and Collaborative Filtering Approach}
    \item \href{https://link.springer.com/chapter/10.1007/978-3-031-59707-7_9}{Sentiment Analysis and Innovative Recommender System: Enhancing Goodreads Book Discovery Using Hybrid Collaborative and Content Based Filtering}
    \item \href{https://link.springer.com/chapter/10.1007/978-3-031-34622-4_13}{Integrated Music Recommendation System Using Collaborative and Content Based Filtering, and Sentiment Analysis}
    \item \href{https://link.springer.com/chapter/10.1007/978-981-99-2287-1_50}{Personalized Recommendation of Literature Resources in University Library Based on Abstract Content Filtering Algorithm}
    \item \href{https://link.springer.com/chapter/10.1007/978-981-16-2594-7_47}{Job Recommendation System Using Content and Collaborative-Based Filtering} 
    \item \href{https://link.springer.com/chapter/10.1007/978-981-19-7660-5_8}{Development of Hybrid Personalized E-commerce Using Collaborative Filtering and Content-Based Filtering for South Cartel Clothing Company}
    \item \href{https://link.springer.com/chapter/10.1007/978-3-031-19958-5_98}{A Hybrid Hotel Recommendation Using Collaborative, Content Based and Knowledge Based Approach}
\end{enumerate} %TODO : temp, only for development
\begin{abstract}
    \ldots
\end{abstract}
\section{Introduction}\label{sec:introduction}

\subsection{Definition}% Definition of Content-Based Filtering | DRAFT, TODO : change the name

\subsection{Importance in recommendation systems} %TODO

\subsection{Overview} %TODO
\section{Types of content-based filtering}\label{sec:cbf_types}

\subsection{Text-based CBF} %TODO

\subsection{Multimedia-based CBF} %TODO
\section{Data representation in content-based filtering}\label{sec:cbf_data_rep}

\subsection{Data extraction} %TODO

\subsection{Data unification using embeddings} %TODO

\subsection{Advantages of embeddings in CBF} %TODO
\section{Advantages and disadvantages of content-based filtering}\label{sec:cbf_advantages_disadvantages} %TODO : could be rewrited to usecases of CBF

\subsection{Advantages} %TODO

\subsection{Limitations} %TODO

 %TODO : temp, could be removed
\section{Comparison of CBF with other recommendation methods}\label{sec:cbf_comparison}

\section{Advantages and disadvantages} %TODO : temp, could be removed

\subsection{Collaborative filtering vs. content-based filtering} %TODO : rewrite name

\subsection{Hybrid Models} %TODO
\section{Use cases}\label{sec:usecases} % TODO : change the name

\subsection{Media streaming platforms} %TODO

\subsection{News} %TODO

\subsection{E-commerce} %TODO
\section{Conclusion}\label{sec:conclusion}

\nocite{*} %TODO : temp

\bibliography{bibliography/references}
\bibliographystyle{plain}
\end{document}
