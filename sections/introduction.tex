\section{Introduction}\label{sec:introduction}
Nowadays, internet resources are overfilled with large amounts of information, within which the user needs to be provided with relevant data to his interests. Without the help of the environment, users could not analyze this mass of information, which reduces the amount of processed data\cite{InternetBigData}.

Recommendation algorithms become a solution to this problem, because they make it possible to select individual content for each user, depending on their actions or the actions of other users. This approach improves the quality of information provided to the user making it more relevant to user preferences\cite{IMRSUCBF}.

Recommendation systems are used in the majority of social networks. Youtube uses different algorithms to recommend content that best matches users' preferences. Linkedin provides flexible jobs searching, Netflix suggests movies based on previous user actions\cite{CBF_In_Social_Networks}.

Content-based filtering (CBF) is an instrument, which is based on finding similarities between user and content. At the same time, the user`s previous actions as well as their profile and the other information, can be taken into account. CBF can be used with different types of data, such as text, audio and images\cite{van2000using}.

Due to the flexibility of Content-based filtering, it is highly used in many approaches. With the use of this approach, recommendation systems provide more relevant data to user preferences, because this is the main idea of CBF. Different systems use different algorithms and types of Content-based filtering, based on the needs and structure of provided data. %TODO : provide a cite if needed

Content-based filtering could be a part of hybrid recommendation systems, which use multiple recommendation approaches to select relevant data and minimize disadvantages of each method. In this case, not only individual properties of CBF, but also a new combination of of this method and some others must be considered\cite{hybrid_systems}.
