\begin{abstract}
    In recent years, recommendation systems have become an essential part of different platforms from e-commerce to the majority of streaming platforms. Content-based filtering is a widely used approach that recommends new items based on user previous interactions and preferences. This article will consider the advantages and disadvantages of content-based filtering, different algorithms for CBF, the connections between this method and other types of recommendation algorithms, the advantages and disadvantages of each algorithm and in which situations it is better to use each of them. Additionally, the article provides an analysis of popular Content-based filtering methods efficiency based on a dataset of books and user interactions.
\end{abstract}